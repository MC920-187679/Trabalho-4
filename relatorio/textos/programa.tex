\section{O Programa}

Todas as ferramentas foram desenvolvidas e \red{testadas} em Python 3.6 ou superior. Além da biblioteca padrão da linguagem, foram utilizados também os pacotes OpenCV, para entrada e saída de imagens, e Numpy \red{1.17}, para aplicação vetorizada da esteganografia.

\subsection{Código-Fonte}

    Neste trabalho foram elaboradas duas ferramentas, \texttt{codificar.py} e \texttt{decodificar.py} que fazem todo o processo de esteganografia. Boa parte do código delas é compartilhado pelos arquivos na pasta \texttt{lib}, como apresentado a seguir.

    \begin{description}
        \item[codificar.py] Ferramenta de codificação de um arquivo em um imagem.

        \item[decodificar.py] Ferramenta de recuperação do arquivo escondido.

        \item[lib] Pacote compartilhado pelas ferramentas.

        \begin{description}
            \item[bits.py] Serialização e processamento de dados em vetores de bits.

            \item[estego.py] Ocultação e recuperação de dados em imagens.

            \item[permuta.py] Reposicionamento pseudo-aleatório dos dados binários.

            \item[args.py] Processamento dos argumentos da linha de comando.

            \item[inout.py] Tratamento de entrada e saída do programa.

            \item[tipos.py] Tipos para checagem estática com MyPy.
        \end{description}
    \end{description}

    Todas as imagens base para o processamento discutido ao longo do texto estão presente na pasta \texttt{imagens}. Os textos de exemplo da esteganografia estão na pasta \texttt{textos}.

    Também existem dois \textit{scripts} em Python utilizados para análise dos resultados: \texttt{plano.py}, para extração de planos de bit, e \texttt{dist.py}, para medidas de similaridade de imagens. Por fim, o \textit{script} \texttt{run.sh} em Bash refaz todos resultados apresentados neste relatório.

\subsection{Execução}

    A execução de ambos os programas deverá ser feita através do interpretador de Python 3.6+. Os exemplos de execuções a seguir funcionam apenas em Python 3.7+, mas o \textit{script} \texttt{run.sh} também funciona em 3.6.

    Todas as opções a seguir também estão explicadas com o texto de ajuda de cada ferramenta. Ele pode ser acessado com a \textit{flag} \mintinline{bash}{--help} ou apenas \mintinline{bash}{-h}.

    \subsubsection{Codificação}

    A primeira tem um argumento obrigatório: a imagem, preferencialmente PNG, onde os dados serão ocultados. Um segundo argumento pode ser passado, especificando o arquivo a ser armazenado. Caso ele não apareça, o arquivo é lido da entrada padrão.

    Por padrão, a imagem resultante é apresentada em uma nova janela gráfica. No entanto, a imagem pode ser salva com a opção \mintinline{bash}{--output} ou \mintinline{bash}{-o} e o arquivo de salvamento. A seguir temos a exemplo de como guardar o texto ``MC920'' na imagem \texttt{baboon.png} com o resultado salvo em \texttt{saida.png}.

    \begin{minted}{bash}
        $ echo MC920 | python3 codificar.py imagens/baboon.png -o saida.png
    \end{minted}

    Note que o arquivo ocultado pode ser qualquer arquivo, não apenas texto. Além disso, por padrão o arquivo é escrito no plano de bit menos significativo da imagem. Isso pode ser alterado com a opção \mintinline{bash}{--bit} ou \mintinline{bash}{-b}. Por exemplo:

    \begin{minted}{bash}
        $ python3 codificar.py imagens/watch.png imagens/monalisa.png -b 2
    \end{minted}

    Existe ainda uma opção sobre a ordem de escrita na imagem, \mintinline{bash}{-p} ou \mintinline{bash}{--permuta}, que será melhor apresentada na seção \red{QUAL\qm}.

    \subsubsection{Decodificação}

    A ferramenta de decodificação é mais simples, recebendo apenas a imagem já esteganografada. Após a decodificação, o arquivo resultante é escrito na saída padrão. Um segundo argumento pode ser especificado para armazenar o resultado em um arquivo.

    Note que, para funcionamento correto da ferramenta, é necessário que a decodificação seja feita no mesmo plano de bit em que o arquivo doi embutido. Para isso, existe a \textit{flag} \mintinline{bash}{--bit} ou \mintinline{bash}{-b}, assim como na codificação. Por exemplo:

    \begin{minted}{bash}
        $ echo MC920 | python3 codificar.py imagens/peppers.png -o saida.png -b 1
        $ python3 decodificar.py saida.png -b 1
        MC920
    \end{minted}
