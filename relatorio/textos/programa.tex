\section{O Programa}

Neste trabalho foram desenvolvidas duas ferramentas, \texttt{codificar.py} e \texttt{decodificar.py} que fazem todo o processo de esteganografia. Boa parte do código delas é compartilhado pelos arquivos na pasta \texttt{lib} do código-fonte. Além disso, também existem duas ferramentas usadas para análise dos resultados, a \texttt{dist.py} e \texttt{plano.py}, que não são necessárias para a execução do \red{processo de esteganografia}.

Todas as ferramentas foram desenvolvidas e \red{testadas} em Python 3.6 ou superior. Além da biblioteca padrão da linguagem, foram utilizados também os pacotes OpenCV, para entrada e saída de imagens, e Numpy \red{1.17}, para aplicação vetorizada da esteganografia.

\subsection{Código-Fonte}

    \begin{description}
        \item[codificar.py] Pacote interno com as operações de limiarização.

        \item[decodificar.py] Pacote interno com as operações de limiarização.

        \item[lib] Pacote interno com as operações de limiarização.

        \begin{description}
            \item[bits.py] Wrapper para a chamada das funções em C.

            \item[estego.py] Funções de cálculo do limiar da vizinhança do pixel.

            \item[permuta.py] Operações de mínimo, máximo, média, desvio padrão e mediana.

            \item[args.py] Tratamento de entrada e saída do programa.

            \item[inout.py] Tratamento de entrada e saída do programa.

            \item[tipos.py] Tipos para checagem estática com MyPy.
        \end{description}

        \item[plano.py] Pacote interno com as operações de limiarização.

        \item[dist.py] Pacote interno com as operações de limiarização.
    \end{description}
