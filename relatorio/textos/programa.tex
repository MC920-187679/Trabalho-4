\section{O Programa}

Todas \red{QUAIS} as ferramentas foram desenvolvidas e \red{testadas} em Python 3.6 ou superior. Além da biblioteca padrão da linguagem, foram utilizados também os pacotes OpenCV, para entrada e saída de imagens, e Numpy \red{1.17}, para aplicação vetorizada da esteganografia.

\subsection{Código-Fonte}

    Neste trabalho foram desenvolvidas duas ferramentas, \texttt{codificar.py} e \texttt{decodificar.py} que fazem todo o processo de esteganografia. Boa parte do código delas é compartilhado pelos arquivos na pasta \texttt{lib} do código-fonte, como apresentado a seguir.

    \begin{description}
        \item[codificar.py] Ferramenta de codificação de um arquivo em um imagem.

        \item[decodificar.py] Ferramenta de recuperação do arquivo escondido.

        \item[lib] Pacote compartilhado pelas ferramentas.

        \begin{description}
            \item[bits.py] Serialização e processamento de dados em vetores de bits.

            \item[estego.py] Ocultação e recuperação de dados em imagens.

            \item[permuta.py] Permutação e embaralhamento de dados binários.

            \item[args.py] Processamento dos argumentos da linha de comando.

            \item[inout.py] Tratamento de entrada e saída do programa.

            \item[tipos.py] Tipos para checagem estática com MyPy.
        \end{description}
    \end{description}

    Todas as imagens base para o processamento discutido ao longo do texto estão presente na pasta \texttt{imagens}. Os textos de exemplo da esteganografia estão na pasta \texttt{textos}.

    Também existem dois \textit{scripts} em Python utilizados para análise dos resultados: \texttt{plano.py}, para extração do plano de bit, e \texttt{dist.py}, para medidas de similaridade de imagens. Por fim, o \textit{script} \texttt{run.sh} em Bash refaz todos resultados apresentados neste relatório.