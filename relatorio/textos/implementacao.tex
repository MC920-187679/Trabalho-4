\section{Implementação}

O primeiro passo no algoritmo \red{codificação} implementado para esteganografia foi transformar o dados do arquivo de entrada em um vetor de bits. Assim, a mensagem ``IC'', dada pelos bytes 0x49 e 0x43, seria representada pelo vetor $[1, 0, 0, 1, 0, 0, 1, 0, 1, 1, 0, 0, 0, 0, 1, 0]$. Para que seja possível a recuperação, também é adicionado um vetor de 64 bits antes do vetor de bits da imagem, representando o tamanho desse vetor.

Para injetar esse vetor na imagem são feitas algumas manipulações binárias simples. Para cada byte da imagem é aplicada uma máscara responsável por zerar o plano de bit especificado. Após isso, o bit que será inserido é deslocado (por \textit{bitshift}) para a posição certa e aplicado por um \textit{bitwise or} no byte da imagem novamente. Isso é feito até encerrar o vetor de bits

No código, isso foi feito de forma vetorizada, utilizando uma matriz booleana para indexação.

Para a decodificação o processo é ainda mais simples. Basta extrair o plano de bit correto em um vetor de bits, recupera o tamanho do arquivo nos primeiros 64 bits e usa isso para recuperar o vetor de bits que realmente fazem parte do arquivo. O resultado é uma mera aglutinação desses bits em caracteres ou bytes.

\subsection{Permutação} % TODO: esse nome?

O modelo anterior insere os bits sempre do começo pro final da imagem, o que pode gerar marcas bem claras na imagem \red{(resultados\qm)}. Uma maneira de evitar isso é distribuindo os dados do arquivo por toda a imagem, como é feito na ferramenta com a opção \mintinline{bash}{--permuta} ou \mintinline{bash}{-p}.

Nesse método basicamente é feita uma permutação da matriz booleana de indexação, inserindo os dados em posições mais distribuídas \red{pela} imagem. Para evitar o surgimento de algum padrão de bits, os dados da imagem também são permutados antes da inserção.

De forma a tornar o método reversível, mas imprevisível, em cada execução é gerada uma chave aleatória. Essa chave é usada como semente para o gerador pseudo-aleatório, sendo depois inserida nos últimos 64 bits do plano especificado na imagem. Todo esse método de permutação funciona apenas com a versão 1.17 ou superior de Numpy.

Com o tamanho fixado nos primeiros 64 bits e a chave nos últimos 64 bits, os dados dispersos na imagem podem ser facilmente recuperados, revertendo o processo anterior. Para indicar se houve a permutação na esteganografia ou não, a ferramenta usa o primeiro bit do plano de bits, efetivamente reduzindo o tamanho representável para 63 bits. Assim, a ferramenta de decodificação consegue recuperar os dois métodos, sem necessidade de indicação do usuário.
