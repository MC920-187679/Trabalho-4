\section{Conclusão}

Pelos resultados da seção anterior, podemos ver que a esteganografia é bem útil quando o objetivo é comunicação sem chamar atenção. Usando os planos de bits menos significativos das imagens é quase impossível que um humano a olho nu suspeite de alguma coisa.

No entanto, algum programa ou programador buscando algo estranho, pode facilmente achar no plano de bit certo. Isso pode ficar bem claro, especialmente com mensagens ou arquivo textuais. Nesse caso, a redistribuição dos dados, discutida na \cref{sec:permutacao}, ou a criptografia dos dados pode acabar ajudando.

Também para arquivos de texto, os traços normalmente são bem marcantes. Isso é evitável também com algum tipo de compressão, como \textit{zip}, que também aumenta a capacidade de armazenamento ou transmissão.
