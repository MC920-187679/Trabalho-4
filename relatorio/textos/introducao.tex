\section{Introdução} \label{sec:introducao}

Esteganografia é uma técnica de comunição secreta, onde a mensagem é escondida em parte de outro texto ou objeto. Dessa forma, apenas o remetente e o destinatário saberiam onde encontrar a mensagem oculta, podendo se comunicar em sigilo.

Isso é diferente da criptografia em que a mensagem continua visível, mas sem sentido algum, exceto para os comunicadores, que conseguem descriptografar a mensagem. Normalmente, a criptografia acaba sendo aplicada na esteganografia, para garantir maior segurança no processo.

Neste trabalho, foram implementados algoritmos simples de esteganografia em imagens digitais, com objetivo de analisar os casos de uso e os limites desse processo. Além disso, foram discutidos alguns vestígios comuns deixados na esteganografia e possíveis soluções para esses problemas, como o posicionamento dos dados de entrada em posições semi-aleatorizadas.
